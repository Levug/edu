\documentclass{article}
\usepackage{empheq}
\usepackage[a4paper, left=20mm, right=20mm, top=20mm, bottom=30mm]{geometry}


\begin{document}

\textit{\textbf{Tunneling and Electric-Field Effects on Electron-Hole Localization
in Artificial Molecules}}//
As a first step, I compute the single-particle eigenfunctions and eigenvalues for electrons
and holes within the envelope-function and effective-mass approximations [1]. The external
2
confinement of the carriers in the double dot is described by a prototypical confinement
potential, which is double-box-like along z and parabolic in the (x, y)-plane; an additional
term in the single-particle Hamiltonian $H{SP}$ accounts for the electric field F, directed along
z:


$1·  H{SP}=\sum{\alpha=e,h}^{} \int {\psi^{\dagger}{\alpha}}({\textbf{r}}){\Bigg[-\frac{{\hbar}^{2}}{{2m}^{}{\alpha}}{\nabla}^{2}+ {V}^{DW}{\alpha}(z) ... \\ ... \frac{1}{2} {m}^{}{\alpha} {\omega}^{2}{\alpha} ({x}^{2}+{y}^{2})-{q}^{}{\alpha}Fz \Bigg]{\psi}^{}{\alpha}({\textbf{r}})d({\textbf{r}})}.$

 $1· X^3 /(2·3) +  ... \\ ...+ 1·3·...·(2·N –1)· X^{2·N+1}.$


where

\end{document}

\frac{\hslash }{} 